\documentclass[11pt]{article}
\usepackage[T1]{fontenc} %thanks's daleif
\usepackage[utf8]{inputenc}
\usepackage[english, danish]{babel}
\usepackage{amsmath,amssymb,amsthm}
\usepackage{graphicx}
\usepackage{mdframed}
\usepackage{mathtools}
\usepackage{float}
\graphicspath{ {./figures/} }

\usepackage[width=17cm,headheight=12cm, bottom=3cm]{geometry}
\usepackage{fancyhdr}
\usepackage{lastpage}
\usepackage{derivative}
\setlength\parindent{0pt}
% =========

\newcommand{\R}[1][]{\ensuremath{\mathbb{R}^{#1}}}

\newtheorem*{theorem}{Theorem}
\newtheorem*{lemma}{Lemma}
\newtheorem*{corollary}{Corollary}
\newtheorem*{Definition}{Definition}
\newenvironment{proposition}
{
    \paragraph{Proposition:}
}
{}


\newenvironment{opgave}[1]
{
    \begin{samepage}
    \begin{mdframed}
    {\bfseries Opgave #1 \\}
}
{
    \end{mdframed}
    \end{samepage}
}

\newenvironment{svar}[1]
{{\bfseries Svar #1} \\}
{\begin{flushright}
    $ \square $
\end{flushright} }

% ========== Header and footer
\pagestyle{fancy}
\fancyhf{}
\rhead{Af: Asbjørn Nøhr Dissing, s224033}
\lhead{KURSUS, \today}
\rfoot{Side \thepage \hspace{1pt} af \pageref{LastPage}}

\begin{document}

% ========== Edit your name here
\author{Asbjørn Nøhr Dissing\\s224033}
\title{Opgavenavn}
\maketitle

\newpage
% ========== Begin answering questions here
\begin{opgave}{S2.3}
Sæt $\mathcal{J}=\{[a, a+1) \mid a \in \mathbb{R}\} .$ Vis at $\sigma(\mathcal{J})=\mathcal{B}(\mathbb{R})$.
\end{opgave}
Vi ser at $ \sigma(\mathcal{J}) \subseteq \mathcal{B} (\mathbb{R}) $ gælder trivielt, siden $ \mathcal{J} $ blot er alle de halvåbne intervaller med længde 1. Per \textbf{Theorem 3.7} ser vi netop at $ \mathcal{J} \subseteq \mathcal{B} (\mathbb{R})$, vi konkluderer fra \textbf{Remark 3.5} at $ \sigma(\mathcal{J}) \subseteq \sigma(\mathcal{B} (\mathbb{R})) = \mathcal{B} (\mathbb{R}) $. Vi viser at $ \mathcal{B} (\mathbb{R}) \subseteq \sigma(\mathcal{J}) $. Hvis vi betragter mængden af halvåbne intervaller $ \mathcal{F} $ i $ \mathbb{R} $, så kan vi vise at
\begin{align*}
    I \in \mathcal{F} \implies I \in \sigma(\mathcal{J})
\end{align*}




\end{document}